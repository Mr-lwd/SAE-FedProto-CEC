\documentclass[journal]{IEEEtran}
\usepackage{amsmath,amsfonts}
% \usepackage{algorithmic}
\usepackage{algorithm}
\usepackage{algpseudocode} % For more flexible 
\usepackage{array}
\usepackage[caption=false,font=normalsize,labelfont=sf,textfont=sf]{subfig}
\usepackage{textcomp}
\usepackage{stfloats}
\usepackage{url}
\usepackage{verbatim}
\usepackage{graphicx}
% \usepackage{booktabs} % For professional-quality tables
\usepackage{cite}

\usepackage{makecell}   % 支持更灵活的单元格对齐和内容换行
\hyphenation{op-tical net-works semi-conduc-tor IEEE-Xplore}
% updated with editorial comments 8/9/2021

\begin{document}

\title{Semi-Asynchronous Energy-Efficient Federated Prototype Learning for Client-Edge-Cloud Architectures}

% \author{IEEE Publication Technology,~\IEEEmembership{Staff,~IEEE,}

% The paper headers
% \markboth{Journal of \LaTeX\ Class Files,~Vol.~14, No.~8, August~2021}%
% {Shell \MakeLowercase{\textit{et al.}}: A Sample Article Using IEEEtran.cls for IEEE Journals}

% \IEEEpubid{0000--0000/00\$00.00~\copyright~2021 IEEE}
% Remember, if you use this you must call \IEEEpubidadjcol in the second
% column for its text to clear the IEEEpubid mark.

\maketitle

\begin{abstract}

\end{abstract}

\begin{IEEEkeywords}
  Federated Prototype Learning, Hierarchical Architecture, Heterogeneous Models, Asynchronous Communication, Energy Efficiency
\end{IEEEkeywords}

\section{Introduction}
\IEEEPARstart{T}{his}

\section{Related Work}

\section{Motivation}

\section{Method}
\subsection{Proposed Algorithm New Version}
\begin{table}[H]
  \caption{Symbol Table}
  \centering
  \renewcommand{\arraystretch}{1.2}
  \begin{tabular}{|@{}m{1cm}<{\centering}|m{6.5cm}|}
    % \toprule
    \hline
    \textbf{Symbol}                       & \textbf{Description}                                                                                                                  \\
    % \midrule
    \hline
    \( J \)                               & Number of classes                                                                                                                     \\
    \hline
    \( T \)                               & Global communication rounds                                                                                                                         \\
    \hline
     \( E \)                               & Edge communication rounds                                                                                                                           \\
    \hline
     \( K \)                               & Local train epochs                                                                                                                          \\
    \hline
     \( L \)                               & Number of edge servers                                                                                                                \\
    \hline
     \( B \)                               & The buffer of the cloud server with a static length                                                                                   \\
    \hline
     \( N^l \)                             & The number of clients on the $l$-th each edge server                                                                                      \\
    \hline
    \(\mathcal{N}_j^l\) & The number of instances belonging to class \( j \) on edge \( l \) that have already participated in computation                                           \\
    \hline
    \( {\mathcal{N}_j^{l,\text{prev}}}  \) & Previous number of instances belonging to class \( j \) on edge that have already participated in computation $l$                               \\
    \hline
    \(\mathcal{N}_j^{l, i}\) & The number of instances belonging to class \( j \) on client \( i \) on edge \( l \) \\ 
    \hline
    % \( \mathcal{N}_j  \)           & Number of edge servers $l$ containing class $j$ that have participated in aggregation                            \\
    % \( {\mathcal{N}_j}^{'}  \)           & Previous number of edge servers $l$ containing class $j$ that have participated in aggregation                            \\

    % \( S^l \)        & Set of clients in the \( l \)-th edge server  \\ 
    \( S^{l} \)                           & Set of clients participating in training in the \( l \)-th edge server                                                                \\
    \hline
    \( \bar{C}_j \)                       & Aggregated prototype of class $j$ in the cloud edge server                                                                            \\
    \hline
    \( C^l_j \)                           & Aggregated prototype of class $j$ from the \( l \)-th edge server                                                                     \\
    \hline
    \( C^{l,\text{prev}}_j \)              & Previous version of aggregated prototype of class $j$ from the \( l \)-th edge server in the cloud server                          \\
    \hline
    \( c^l_{i, j} \)                      & Aggregated prototype of class $j$ from the client \(i\) in  the \( l \)-th edge server                                                \\
    \hline
    \( c_{i, j}^{l, \text{prev}} \)        & Previous version of the aggregated prototype of class \( j \) from client \( i \) in the \( l \)-th edge server in the edge server \\
    \hline
    \( D^l_{i,j} \)                         & A subset of the local dataset \(D^l_i\) of the $i$-th client in the $l$-th edge server, containing training instances of class $j$.                               \\
    \hline
    \( \mathcal{X}^l_i \) & The feature and label set of the $i$-th client in the $l$-th edge server, containing all features of $D_i$. \\
    \hline
    \( G^l \) & The global classifier in the $l$-th edge server. \\
    % \bottomrule
    \hline
  \end{tabular}
\end{table}

The following algorithm demonstrates how to calculate the factorial of a number.

\begin{algorithm}[H]
  \caption{Hierarchical Federated Prototype Learning -Part 1}
  \begin{algorithmic}[1]
    % \State \textbf{Input: nothing}
    % \State \textbf{Output: nothing}

    \Procedure{cloud server executes}{}
    % \State Initialize global prototype set \( \bar{C}\) for all classes and weights for clients with heterogeneous models.
    \State Initialize weights for clients with heterogeneous models.
    \State All edge servers execute in parallel.
    \For{$t = 1, \dots, T$}
    \State Clear the buffer \(B\)

    \While{$B$ is not full} \Comment{Async process}
    \State Receive a triple \( (C^l, \mathcal{N}^l) \) from one edge server.
    \State Populate \( B \) with the received triple.
    \EndWhile
    \State \( \bar{C}, G \gets \text{CloudUpdate}(B) \)
    \State Send \( \bar{C}, G \) to edge servers participating in the current global aggregation.
    \State These edge servers re-execute.
    \EndFor
    \EndProcedure

    \Procedure{edge server executes}{}
    \State Receive \( \bar{C}, G \) from the cloud server
    \State Choose a set of clients $S^l$ to train in parallel.
    \For{$e = 1, \dots, E$} \Comment{E now is static 1}
    \State Send \( \bar{C}, G \) to client \( i \in S^{l} \)
    \For{each client \( i \) in parallel}
    \State \( c^l_i \gets \text{ClientUpdate}(i, \bar{C}, G) \)
    \EndFor \Comment{Wait for all clients}
    \State \( (C^l, \mathcal{N}^l) \gets \text{EdgeAggregate}(\{ c^l_i \}_{i \in S^{l}}) \)
    \State \( \bar{C} \gets \text{EdgeUpdate}(\bar{C}, C^l) \) \Comment{not used now}
    \EndFor
    \State Send a triple \( (C^l, \mathcal{N}^l) \) to the cloud server
    \EndProcedure
  \end{algorithmic}
\end{algorithm}

% \newpage  % 强制换页

\begin{algorithm}[H]
  \caption{Hierarchical Federated Prototype Learning -Part 2}
  \begin{algorithmic}[1]
    \Procedure{CloudUpdate}{B}
    \For{$j = 1, \dots, J$}
    \State \( \hat{C}_{j} \gets \sum_{l=1}^{L} {\mathcal{N}_j^{l,\text{prev}}} \cdot\bar{C}_{j} \) \Comment{Extend the aggregated prototypes $\bar{C}$}

    \For{$(C^l, \mathcal{N}^l) \in B$}
    \State \( \hat{C}_{j}  \gets \hat{C}_{j} + {\mathcal{N}_j^l} \cdot C^l_j \)
    \If{$C^{l,\text{prev}}_{j}$ is not empty}
    \State \( \hat{C}_{j}  \gets \hat{C}_{j} - {\mathcal{N}_j^{l,\text{prev}}} \cdot C_j^{l, \text{prev}} \)
    % \State \( \mathcal{N}_j \gets \mathcal{N}_j + 1 \)
    \EndIf
    \State \( {\mathcal{N}_j^{l,\text{prev}}} \gets \mathcal{N}_j^l \)
    \EndFor
    \State \( \bar{C}_{j} \gets \frac{\hat{C}_{j}}{\sum_{l=1}^{L} {\mathcal{N}_j^{l,\text{prev}}}} \)
    \EndFor
    \State \( C^{l,\text{prev}} \gets C^l \) for \( l \in B \)
    \For{$j = 1, \dots, J$}
    \State \(\mu_j \gets \bar{C}_j\)
    \State \(\Sigma_j = \frac{1}{\mathcal{N}_j-1} \sum_{l=1}^{L} {\mathcal{N}_j^{l,\text{prev}}}(C^{l,\text{prev}} - \mu)(C^{l,\text{prev}} - \mu)^\top\)
    \EndFor

    \State Generate samples \(X \sim \mathcal{N}(\mu, \Sigma)\) using the multivariate normal distribution
    \State Use the generated samples \(X\) to train the global classifier \( G \) 
    \State \Return \( \bar{C}, G \)
    \EndProcedure

    \Procedure{EdgeAggregate}{$l, \{ c^l_i \}_{i \in S^{l}}$}
    \For{$j = 1, \dots, J$}
    \State \( {\hat{C^l_j}} \gets \mathcal{N}^l_j \cdot C^l_{j} \)
    
    \For{each $c^l_i$}
    \State \( {\hat{C^l_j}}  \gets {\hat{C^l_j}} +  \mathcal{N}_j^{l, i} \cdot c^l_{i,j} \)
    \If{$c^{l,\text{prev}}_{i,j}$ is not empty}
    \State \( {\hat{C^l_j}}  \gets {\hat{C^l_j}} - \mathcal{N}_j^{l, i} \cdot c^{l,\text{prev}}_{i,j}\)
    \Else
    \State \( \mathcal{N}_j^l \gets \mathcal{N}_j^l + \mathcal{N}_j^{l, i} \)
    \EndIf
    \EndFor
    \State \( {C^l_j} \gets \frac{{\hat{C^l_j}}}{\mathcal{N}_j^l} \)
    \EndFor
    \State \( c^{l,\text{prev}}_i \gets c^l_i \) for \( i \in S^l \)

    \State \Return \( (C^l, \mathcal{N}^l) \)
    \EndProcedure

    \Procedure{ClientUpdate}{$i, \bar{C}, G$}
    \State Receive \( \bar{C}, G \) from the edge server
    \For{$k=1, \dots, K$}
    \State DVFS to be implemented...
    \For{batch (\(x,y\)) $\in$ \(D_i\)}
    \State Compute client prototypes by Eq.?.
    \State Compute loss by Eq.? using client prototypes and the global classifier $G$.
    \State Update client model according to the loss.
    \EndFor
    \EndFor
    \State \Return \( c^l_i \)
    \EndProcedure
  \end{algorithmic}
\end{algorithm}

\section{Experiments}

\section{Conclusion}


\end{document}


